\documentclass[10pt]{beamer}

\usetheme[progressbar=frametitle]{metropolis}
\usepackage{appendixnumberbeamer}

\usepackage{booktabs}
\usepackage[scale=2]{ccicons}

\usepackage{pgfplots}
\usepgfplotslibrary{dateplot}

\usepackage{xspace}
\newcommand{\themename}{\textbf{\textsc{metropolis}}\xspace}

\usepackage[brazil]{babel}
\usepackage[utf8]{inputenc}

\title{Employers' system}
\subtitle{Gerenciamento de funcionários}
% \date{\today}
\date{}
\author{Bruno M. Barbosa}
\institute{Instituto de Computação}
% \titlegraphic{\hfill\includegraphics[height=1.5cm]{logo.pdf}}

\begin{document}

\maketitle

\begin{frame}{Sumário}
  \setbeamertemplate{section in toc}[sections numbered]
  \tableofcontents[hideallsubsections]
\end{frame}

\section{Visão geral}

\begin{frame}[fragile]{Descrição}

  O sistema em questão trata-se de um simples sistema voltado para o gerenciamnto de cadastro de funcionários em uma empresa.
  
  Feito com o objetivo de exemplificar uma operação de CRUD (Create-Read-Update-Delete) em um sistema Web
  
\end{frame}

\begin{frame}[fragile]{Descrição}
  
  Dispõe de uma tela inicial de navegação, onde é possível selecionar Cadastro inicial e listagem de todos os funcionários, e mais uma, onde é possível alterar dados existentes.
  
\end{frame}

\section{Ferramentas}

\begin{frame}{Ferramentas}
	
	\begin{figure}
	    \centering
	    \includegraphics[scale=.3]{img/angular-rest.jpg}
	\end{figure}
	
\end{frame}


    % \metroset{titleformat frame=smallcaps}
\section{Atualizações}
\begin{frame}{Atualizações}
    
    \begin{itemize}
        \item Acréscimo de estilo CSS
	    \item Utilização do recurso de Routing, do Angular 
    \end{itemize}
	
\end{frame}


\section{Trabalhos Futuros}
\begin{frame}{Trabalhos Futuros}
    
        Acréscimo de uma tela de login, onde cada usuário teria um login e senha, e alteraria seus dados individualmente. 
        Cada um teria permissão de root para alterar tudo, ou normal, para alterar apenas os seus dados.
    
\end{frame}

\begin{frame}{Summary}

  Apresentação e código disponíveis em:

  \begin{center}\url{https://github.com/brunomb91/Employers}\end{center}

\end{frame}

\begin{frame}
  \begin{quote}
    Veni, Vidi, Vici
  \end{quote}
\end{frame}

{\setbeamercolor{palette primary}{fg=black, bg=yellow}
\begin{frame}[standout]
  Obrigado!
\end{frame}
}

\end{document}
