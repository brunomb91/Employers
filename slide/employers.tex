\documentclass[12pt]{beamer}
\usepackage[brazil]{babel}
\usepackage[utf8]{inputenc}


\addtobeamertemplate{navigation symbols}{}{%
    \usebeamerfont{footline}%
    \usebeamercolor[fg]{footline}%
    \hspace{1em}%
    \insertframenumber/\inserttotalframenumber
}

\title{Employers' system}
\author{Bruno Marques Barbosa}
\begin{document}
\maketitle


\begin{frame}{Sumário}
%  \frametitle{Sumário}
  \setbeamertemplate{section in toc}[sections numbered]
  \tableofcontents[currentsection]
\end{frame}

\section{Visão Geral}
\begin{frame}{Visão Geral}
\frametitle{Visão Geral}

\begin{itemize}
\item Aplicação web básica para cadastro de funcionários de uma empresa 
\pause
\item Sistema utilizado como exemplo de CRUD (Create-Read-Update-Delete) em um sistema web
\pause
\item Sistema implementado em modo API
\end{itemize}

\end{frame}

\section{Ferramentas}
\begin{frame}{Ferramentas}
\frametitle{Ferramentas}

\begin{figure}
\includegraphics[scale=.3]{img/angular-rest}
\end{figure}

\end{frame}

\section{Melhorias}
\begin{frame}{Melhorias}
\frametitle{Melhorias}
\begin{itemize}
\item Utilização de \textit{routes} do Angular
\pause
\item CSS
\end{itemize}

\end{frame}

\begin{frame}

\begin{center}
\Huge Execução da aplicação
\end{center}

\end{frame}

\begin{frame}

\begin{center}
\Huge Obrigado!
\end{center}

\end{frame}

\end{document} 